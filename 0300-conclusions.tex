\chapter{Conclusions \& Recommendations}

In conclusion, the ILMS system design specification was successfully created through the use of certain model diagrams and designs. To achieve this, a use case model diagram was created which helped define the interactions between the actors and the system. Additionally, an activity diagram was then developed to accompany the use case model diagram in order to provide details about the justifications for the design decisions. Furthermore, a class model diagram was formed in order to demonstrate the design of the data structure for the system. A functional workflow design was also constructed by formulating a sequence model diagram so that it would represent the dynamic communication of the class diagram. Lastly, a component model diagram was created to illustrate the system component design by representing the reusable patterns of the system’s functions.

We learnt how to take a brief from stakeholders for a system, in this case the ILMS scenario, and then convert it into system requirements. Using these system requirements we learnt how to refine them into a series of diagrams that can be used as tools that assist in the production of the system. As a group, our knowledge of different UML techniques has certainly improved over the duration of this project. Furthermore through creating diagrams working through the system’s context analysis, behaviour analysis, the data model design, functional workflow design and the system’s components design, we were able to effectively see how a problem can be broken down, divided into steps and provide an easily understandable combination of models prepared for a development project. 

In order to make further improvements to this project, professional modelling tools could have been utilised during the model creation process to develop models more quickly and to a more professional standard. This would certainly be preferable to the design tools utilised, such as free browser-based applications such as \href{https://www.draw.io}{draw.io}. Furthermore if attempting this project again, the group would choose to spend more time developing an effective and accurate use case model at the start of the project, considering this took numerous iterations in order to achieve the final product and delayed the development of other diagrams. 